\documentclass{article}
\usepackage{amsthm}
\usepackage[margin=1.0in]{geometry}
\begin{document}

\theoremstyle{mydef}
\newtheorem{mydef}{Definition}[section]
\title{Algebraic Data Models: Theory and Implementation}
\maketitle
\section{Introduction}
This document is describes the mathematical structure and system architecture of a Python project for machine learning using algebraic structures \cite{algebraic}. The project is built on the theoretical foundation established by Mike Izbicki in his work on algebraic classifiers \cite{algebraic classifiers}. The goal of the current project is to make efficient algebraic machine learning models available to Python users, similar to the way that Izbicki's library HLearn \cite{hlearn} makes them available to Haskell users.

\section{Background: Algebraic Vocabulary}
This section includes some basic definitions. Readers already familiar with the definitions of semigroups, monoids, and groups can safely skip this section.
\begin{mydef}
An algebraic structure is a set $S$ and a binary function $\diamond: S \times S \rightarrow S$.

We often refer to S as the "underlying set" of the structure.
\end{mydef}
\begin{mydef}
A semigroup is an algebraic structure where $\diamond$ is associative.
\end{mydef}
\begin{mydef}
A monoid is a semigroup which has an identity.
\end{mydef}
\begin{mydef}
A group is a monoid which has an inverse element for each element in the underlying set.
\end{mydef}
\section{Models with Group structures}

When we learn from data, we create a model of the world. Often, this corresponds to choosing some parametric model over our observations, and then estimating its free parameters by calculating statistics from the data. 

However, it is pointed out in \cite{algebraic classifiers} that models with algebraic structure (particularly models with the properties of groups) can be used to easily construct machine learning algorithms which are fast to cross-validate, easy to parallelize, and naturally implement on-line behavior. This section formalizes these models, in addition to providing useful examples of such models.

From our point of view, these models are just groups with some additional properties. We give the formal definition here:

\begin{mydef}
A group data model $\alpha$ is defined by the triple $(S, \diamond, f)$: 
\begin{itemize}  
    \item A group $(S, \diamond)$
    \item A function $f:D \rightarrow S$ which maps a dataset to the underlying set of the group
	\linebreak 	\linebreak    
    Such that:
    
    \item For any two datasets $D_1$ and $D_2$, $f(D_1) \diamond f(D_2) = f(D_1 \cup D_2)$ 
    \item For any two datasets $D_1$ and $D_2$, $f(D_1) \diamond f(D_2)^{-1} = f(D_1 - D_2)$ 
    
\end{itemize}
\end{mydef}

This could reasonably be called an "algebraic statistic", or a "statistic with algebraic properties" - $f$ is the function which calculates the statistic from a data set. However, we will refer to this as an "algebraic data model" to avoid confusion with the existing field of algebraic statistics.

At this point, we can present a familiar statistic - the sample mean - as a model with a group structure

\begin{mydef}
The sample mean is a group data model defined by the following:
\begin{itemize}
	\item The underlying set S is the real numbers, $R$
	\item The operation is defined by $S_1 \diamond S_2$
	\item 
\end{itemize}
\end{mydef}

In other words

\begin{mydef}
An algebraic density model is:
\begin{itemize}  
    \item A collection of algebraic data models
    \item A PDF
\end{itemize}
\end{mydef}

\begin{mydef}
An algebraic classifier is:
\begin{itemize}  
    \item A collection of algebraic data models
    \item A classification function
\end{itemize}
\end{mydef}


\begin{mydef}
A composite algebraic data model is: 
\begin{itemize}  
    \item A collection of algebraic data models
    \item A collection of functions mapping the model states to some output space
\end{itemize}
\end{mydef}



\section{The Design of the System}

\begin{thebibliography}{9}

\bibitem{algebraic}
  https://github.com/lmc2179/algebraic
  
\bibitem{algebraic classifiers}
  Michael Izbicki,
  \emph{Algebraic Classifiers: a generic approach to fast cross-validation, online training, and parallel training},
    International Conference on Machine Learning,
  2013.

\bibitem{hlearn}
  Michael Izbicki,
  \emph{HLearn: A Machine Learning Library for Haskell},
  Trends in Functional Programming,
  2013.



\end{thebibliography}
\end{document}