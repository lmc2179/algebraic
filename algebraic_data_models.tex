\documentclass{article}
\usepackage{amsthm}
\usepackage[margin=1.0in]{geometry}
\begin{document}
\newtheorem{mydef}{Definition}
\title{Algebraic Data Models: Theory and Implementation}
\maketitle
\section{Introduction}
This document is describes the theoretical underpinnings and system architecture of a Python project for machine learning using algebraic structures \cite{algebraic}. The project is built on the theoretical foundation established by Mike Izbicki in his work on \cite{algebraic classifiers}. The goal of the current project is to make efficient algebraic machine learning models available to Python users, similar to the way that Izbicki's library HLearn \cite{hlearn} makes them available to Haskell users.

\section{Background: Algebraic Vocabulary}
This section includes some basic definitions. Readers already familiar with the definitions of semigroups, monoids, and groups can safely skip this section.
\begin{mydef}
An algebraic structure is a set $S$ and a binary function $\diamond: S \times S \rightarrow S$.
\end{mydef}
\begin{mydef}
A semigroup is an algebraic structure where $\diamond$ is associative.
\end{mydef}
\begin{mydef}
A monoid is an algebraic structure which has an identity.
\end{mydef}
\begin{mydef}
A group is an algebraic structure which has an inverse.
\end{mydef}
\section{The Theoretical Model}

\section{The Design of the System}

\begin{thebibliography}{9}

\bibitem{algebraic}
  https://github.com/lmc2179/algebraic
  
\bibitem{algebraic classifiers}
  Michael Izbicki,
  \emph{Algebraic Classifiers: a generic approach to fast cross-validation, online training, and parallel training},
    International Conference on Machine Learning,
  2013.

\bibitem{hlearn}
  Michael Izbicki,
  \emph{HLearn: A Machine Learning Library for Haskell},
  Trends in Functional Programming,
  2013.



\end{thebibliography}
\end{document}